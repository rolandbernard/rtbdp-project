
\section{\texorpdfstring{Technologies and \\Overall Architecture}{Technologies and Overall Architecture}} \label{architecture}

\subsection{Technology Stack}

\paragraph{Backend Components}
The backend services, i.e., the producer, the frontend server, and the Flink based processor, are implemented in Java \cite{java21}. Apache Maven \cite{mavenDocs} is used for dependency management and as a build system. Both the producer and the frontend server rely on RxJava \cite{rxJavaDocs} for reactive stream processing and managing asynchronous operations. In all backend components the Argparse4j library \cite{argparse4jDocs} is used to handle command-line configuration, while Jackson \cite{jacksonDocs} is utilized for high-performance JSON parsing. Logging is standardized in this project by using SLF4J \cite{slf4jDocs} and Logback \cite{logbackDocs}, with JUnit \cite{junitDocs} used for unit testing to ensure code reliability.

\paragraph{Frontend Client}
The user interface of the project is implemented as a Single Page Application (SPA) built with TypeScript \cite{tsDocs} and React \cite{reactDocs}. Vite \cite{viteDocs} is used to serve as the build toolchain. Styling is handled via Tailwind CSS \cite{tailwindDocs}, and data visualization is powered by the React based Recharts \cite{rechartsDocs} library. Communication with the backend utilizes RxJS \cite{rxjsDocs} to manage WebSocket streams and simple stream operations.

\paragraph{Stream Processing and Storage}
Further, the project makes use of a number of different technologies for the storage, processing, and streaming transmission of data. Firstly, Apache Kafka \cite{kafkaDocs,kreps2011kafka} is used to serve as the central system for data ingestion and messaging in the project. Next, Kafka Flink \cite{flinkDocs,carbone2015apache} serves as the core stream processing engine in the project. In this project it has further been configured to utilize the RocksDB \cite{rocksdbDocs} state backend to manage large state sizes and enable fault tolerance. To provide persistent storage for processed data, the project uses PostgreSQL \cite{postgresDocs}, a popular open-source database management system. The project also uses the TimescaleDB extension \cite{timescaleDocs}, providing facilities for managing time-series data, especially partitioning and retention. Finally, the project makes use of Docker \cite{dockerDocs} and Docker Compose \cite{dockerCompose} to orchestrate all the services.

\subsection{System Architecture}

Lorem ipsum dolor sit amet, consectetur adipiscing elit. Donec nisi risus, suscipit sed mattis non, sagittis sed nisi. Sed a lobortis odio, quis dapibus nisl. Class aptent taciti sociosqu ad litora torquent per conubia nostra, per inceptos himenaeos. Fusce tincidunt sapien eu dolor ullamcorper, quis dapibus lectus venenatis. Nunc a blandit massa. Fusce ac fringilla nibh, nec eleifend mauris. Sed maximus tempus nibh, porta cursus mi rutrum et. Nunc erat massa, commodo id ornare ut, scelerisque ac est. Cras elit metus, aliquet sit amet maximus in, gravida in mauris. Vestibulum ante ipsum primis in faucibus orci luctus et ultrices posuere cubilia curae; Ut enim tellus, ornare quis malesuada non, scelerisque a elit. Donec dictum velit eu dui hendrerit ultrices. Pellentesque posuere porta odio eu feugiat. Phasellus non malesuada libero. Suspendisse sit amet tempus orci, quis dictum tellus. Proin magna massa. 

\subsection{Workflow}

Lorem ipsum dolor sit amet, consectetur adipiscing elit. Donec nisi risus, suscipit sed mattis non, sagittis sed nisi. Sed a lobortis odio, quis dapibus nisl. Class aptent taciti sociosqu ad litora torquent per conubia nostra, per inceptos himenaeos. Fusce tincidunt sapien eu dolor ullamcorper, quis dapibus lectus venenatis. Nunc a blandit massa. Fusce ac fringilla nibh, nec eleifend mauris. Sed maximus tempus nibh, porta cursus mi rutrum et. Nunc erat massa, commodo id ornare ut, scelerisque ac est. Cras elit metus, aliquet sit amet maximus in, gravida in mauris. Vestibulum ante ipsum primis in faucibus orci luctus et ultrices posuere cubilia curae; Ut enim tellus, ornare quis malesuada non, scelerisque a elit. Donec dictum velit eu dui hendrerit ultrices. Pellentesque posuere porta odio eu feugiat. Phasellus non malesuada libero. Suspendisse sit amet tempus orci, quis dictum tellus. Proin magna massa. 

\subsection{Scalability and Reliability}

Lorem ipsum dolor sit amet, consectetur adipiscing elit. Donec nisi risus, suscipit sed mattis non, sagittis sed nisi. Sed a lobortis odio, quis dapibus nisl. Class aptent taciti sociosqu ad litora torquent per conubia nostra, per inceptos himenaeos. Fusce tincidunt sapien eu dolor ullamcorper, quis dapibus lectus venenatis. Nunc a blandit massa. Fusce ac fringilla nibh, nec eleifend mauris. Sed maximus tempus nibh, porta cursus mi rutrum et. Nunc erat massa, commodo id ornare ut, scelerisque ac est. Cras elit metus, aliquet sit amet maximus in, gravida in mauris. Vestibulum ante ipsum primis in faucibus orci luctus et ultrices posuere cubilia curae; Ut enim tellus, ornare quis malesuada non, scelerisque a elit. Donec dictum velit eu dui hendrerit ultrices. Pellentesque posuere porta odio eu feugiat. Phasellus non malesuada libero. Suspendisse sit amet tempus orci, quis dictum tellus. Proin magna massa. 
